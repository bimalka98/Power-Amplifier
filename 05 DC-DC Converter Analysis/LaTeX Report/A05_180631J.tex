\documentclass[a4paper,11pt]{article}%,twocolumn
\input{settings/packages}
\input{settings/page}
\input{settings/jupyter}
\newcommand{\parallelsum}{\mathbin{\!/\mkern-5mu/\!}}
\usepackage[siunitx, RPvoltages]{circuitikz}


\begin{document}
	\begin{center}
		{\large \textbf{ASSIGNMENT 05 – Switching Circuits}}\\
		Thalagala B.P.\hspace{0.5cm} 180631J 
	\end{center}
	\hrule

%small-signal AC behavior
%vin has zero source resistance
%neglect the small-signal output resistances ro1 and ro2 of the transistors Q1 and Q2.
%423 microelectronics Chapter 6 Bipolar Junction Transistors (BJTs)
%6.5.4 Voltage Gain 408

\section*{Q1: $for~0\leq t < DT_s$}

\begin{figure}[!h]
	\centering
\begin{circuitikz}[american] %, voltage shift=-1
	\ctikzset{inductors/scale=1.5, inductor=cute}
	\draw[thick] (0,0)to[V, l=$V_g$] (0,3) to [L, ,l=$L$, i>=$i_{L}$, v=$v_{L}$] (3,3) to[normal closed switch, l=$S_1$, i>=$i_{S_1}$] (3,0) -- (0,0);
	\draw[thick] (3,3) to[capacitor,l=$C$, i>=$i_{C}$,v=$v_{C}$] (6,3);
	\draw[thick] (6,3) to[normal open switch, l=$S_2$,v=$v_{S_2}$] (6,0) --(3,0);
	\draw[thick] (6,3) to[normal closed switch, l=$S_3$, i<=$i_{S_3}$](9,3) ;
	\draw[thick] (9,3) to[capacitor,l=$C_o$, i>=$i_{C_o}$,v=$v_{C_o}$] (9,0)--(6,0);
	\draw[thick](9,3) --(12,3) to[R=$R_o$, v=$V_o$, i>=$I_{o}$](12,0) -- (9,0);
\end{circuitikz}
\caption{Circuit when S1, S3 are turned -ON and	S2 is turned -OFF}
	\label{fc}
\end{figure}

\hrule
\section*{Q2: $for~DT_s \leq t < T_s$}

\begin{figure}[!h]
	\centering
	\begin{circuitikz}[american] %, voltage shift=-1
		\ctikzset{inductors/scale=1.5, inductor=cute}
		\draw[thick] (0,0)to[V, l=$V_g$] (0,3) to [L, ,l=$L$, i>=$i_{L}$, v=$v_{L}$] (3,3) to[normal open switch, l=$S_1$, v = $v_{S_1}$] (3,0) -- (0,0);
		\draw[thick] (3,3) to[capacitor,l=$C$, i>=$i_{C}$,v=$v_{C}$] (6,3);
		\draw[thick] (6,3) to[normal closed switch, l=$S_2$, i>=$i_{S_2}$] (6,0) --(3,0);
		\draw[thick] (9,3) to[normal open switch, l=$S_3$, v = $v_{S_3}$](6,3) ;
		\draw[thick] (9,3) to[capacitor,l=$C_o$, i>=$i_{C_o}$,v=$v_{C_o}$] (9,0)--(6,0);
		\draw[thick](9,3) --(12,3) to[R=$R_o$, v=$V_o$, i>=$I_{o}$](12,0) -- (9,0);
	\end{circuitikz}
	\caption{Circuit when S1, S3 are turned -OFF and S2 is turned -ON}
	\label{sc}
\end{figure}
\hrule
\section*{Q3}
Three main assumptions which can be used to simplify the analysis of a switching	circuit.
\begin{enumerate}
	\item \textbf{Steady state operation}: At each cycle the operation of the circuit is identical.
	\item \textbf{Small ripple approximation}: Therefore $ v_o(t) = V_o + v_{ripple}(t)$ is simplified into $v_o(t) = V_o$ using the fact that $| v_{ripple}(t)| \ll V_o$ which is caused by non-ideal filtering.
%		\item $ i_o(t) = I_o + i_{ripple}(t)$ is simplified in to  $i_o(t) = I_o$ using $| i_{ripple}(t)| \ll I_o$

	\item \textbf{Sufficiently large capacitance}: Throughout the operation capacitors will maintain constant voltages.
\end{enumerate}

\pagebreak
\section*{Q4}

\subsection*{Consider the operation of the circuit for $0\leq t < DT_s$}

\subsubsection*{* Associated Voltages}
Since the $S_1$ switch is ON, positive and negative terminals of the source are directly connected to the positive and negative terminals of the inductor($L$) respectively.
\begin{equation}
	 v_L =  V_g
	 \label{first}
\end{equation}


Since the $S_3$ switch is also ON, positive terminal of the capacitor $C$ is connected to the negative terminals of the capacitor $C_o$ and resistor $R_o$. While negative terminal of the capacitor $C$ is connected to the positive terminals of the capacitor $C_o$ and resistor $R_o$. In addition to that as $C_o$ and $R_o$ are in parallel $v_{C_o} = V_o$.\\

Using the Kirchhoff voltage Low to that loop,
\[	v_C + V_o = 0 \hspace{2cm} \therefore v_C = -V_o \]

%\subsubsection*{* Associated Currents}
%Using Kirchhoff current Low, to the nodes,
%
%\begin{equation}
%	i_L = i_{S_1}+i_C
%\end{equation}
%
%\begin{equation}
%	i_C = -i_{S_3}
%\end{equation}
%
%\begin{equation}
%	-i_{S_3} = i_{C_0} + I_o
%\end{equation}
%
%\[\therefore i_L = i_{S_1} + i_{C_0} + I_o\]

\subsection*{Consider the operation of the circuit for $DT_s \leq t < T_s$}

\subsubsection*{* Associated Voltages}
Using the Kirchhoff voltage Low,
\[ V_g = v_L + v_C \]

Since we assume sufficiently large capacitance, change in $v_C$ is negligible from the previous state of the circuit.
\[\therefore  V_g = v_L + (-V_o) \]
Rearranging,
\begin{equation}
	v_L = V_g + V_o
	\label{second}
\end{equation}

From equations \eqref{first} and \eqref{second},
\begin{equation*}
	v_L = \begin{cases}
		V_g & for~0\leq t < DT_s\\
		V_g + V_o & for~DT_s \leq t < T_s
	\end{cases}
\end{equation*}

As we are analyzing a switching circuit which has the ability to invert and step-up its supply voltage, $|V_g| < |V_o| ~and~ V_o < 0 $. Which implies $V_g + V_o < 0$.

\begin{figure}[!h]
	\centering
	\includegraphics[scale=0.15]{figures/vl}
	\caption{Inductor voltage $v_L(t)$}	
\end{figure}
\pagebreak
Since voltage drop across an inductor is related to the current through it according the following differential equation, graph of the $i_L$ can be directly deduced from the graph of the $v_L$.
\[v_L = L.\frac{d~i_L}{dt}\]

\begin{figure}[!h]
	\centering
	\includegraphics[scale=0.15]{figures/il}	
	\caption{Inductor current $i_L(t)$}
\end{figure}

In the above figure $I_L$ denotes the average inductor current.\\

\hrule
\section*{Q5}
Using inductor volt-second balance,
\[
\begin{split}
	<V_L(t)> = 0 = & \frac{1}{T_s}.\int_{0}^{T_s} V_L(t) ~dt\\
	&=\int_{0}^{DT_s} V_L(t) ~dt + \int_{DT_s}^{T_s}V_L(t) ~dt\\
	&=\int_{0}^{DT_s} V_g  ~dt + \int_{DT_s}^{T_s} V_g + V_o ~dt\\
	&= V_g.DT_s +(V_g + V_o).(T_s - DT_s)\\
	&= V_g.D +(V_g + V_o).(1 - D)\\
	&= V_g + V_o.(1 - D) 
\end{split}
\]

\[\therefore V_o = -\frac{V_g}{1-D}\]

\hrule
\section*{Q6}


\subsection*{S1}
\begin{itemize}
	\item ON Current($i_{S_1}$)	
	
	As described below in the part ``\textbf{S3}'', $i_{S_3}$ is positive in the marked direction. Therefore according to the Kirchhoff Current Low, $i_C + i_{S_3} = 0$ and it implies $i_C$ is negative in the marked direction(actual direction is the opposite of the marked direction). As $v_L$ is positive $i_L$ is positive in the marked direction. Because of these reasons, according to the K.C.L, $i_{S_1} = i_L+(-i_C)$ and it is positive in the marked direction in Fig.\ref{fc}.
%	From Fig.\ref{fc}, using Kirchhoff Current Low, $i_L = i_C + i_{S_1}$. From the calculations of the  Question 04, $v_C = -V_o$. Since $V_o$ is negative, $v_C$ is positive and therefore $i_C$ is positive. While $S_1$ is ON, $v_L = V_g >0$ and therefore $i_L$ is positive. Consider,  $i_L -i_C =  i_{S_1}$
	
	
	\item Blocking Voltage($v_{S_1}$)
	
	From Fig.\ref{sc}, positive and negative terminals of the switch are connected to the positive and negative terminals of the capacitor $C$. Therefore $v_{S_1} = v_C$. 	From the calculations of the  Question 04, $v_C = -V_o$ and $V_o$ is negative. Therefore $v_{S_1}$ is positive.
	
\end{itemize}


\subsection*{S2}
\begin{itemize}
	\item ON Current($i_{S_2}$)	
	
 When considering the loop consist of the Source,inductor, capacitor($C$) and the switch $S_2$ in Fig.\ref{sc}, it is obvious that $i_{S_2}$ is positive in the given direction.
	
%	From the calculations of the  Question 04, $v_C = -V_o$. Since $V_o$ is negative, $v_C$ is positive and therefore $i_C$ is positive. As $i_C = i_{S_2}$ from the Kirchhoff current low(Fig. \ref{sc}),  $i_{S_2}$ is also positive.
	
	
	\item Blocking Voltage($v_{S_2}$)
	
	From Fig.\ref{fc}, $v_{S_2} = v_{C_o} = V_o $. Since $V_o$ is negative, $v_{S_2}$ is also negative.
\end{itemize}


\subsection*{S3}
\begin{itemize}
	\item ON Current($i_{S_3}$)	
	
	Since $V_o~and ~v_{C_o}$ are negative in the marked direction, their currents are also in the direction opposite to the marked direction. Therefore according to the Kirchhoff Current Low, $i_{S_3} = -(i_{C_o} + I_o)$. Which results $i_{S_3}$ to be positive in the marked direction in Fig.\ref{fc}.
%	From Fig.\ref{fc}, using Kirchhoff Current Low, $i_C + i_{S_3} = 0~\therefore i_{S_3} = -i_C$. Since $i_{C}$ is positive in the marked direction, $i_{S_3}$ is positive in the marked direction.
	
	\item Blocking Voltage($v_{S_3}$)
	
	From Fig.\ref{sc}, $v_{S_3} = V_o$ as positive and negative terminals of the switch are connected to the positive and negative terminals of the $R_o$ respectively. Since $V_o$ is negative, $v_{S_3}$ is also negative.
	
\end{itemize}

\begin{table}[!h]
	\centering
	\begin{tabular}{c c c c}
		\textbf{Switch} & \textbf{ON-current} & \textbf{Blocking Voltage} & \textbf{Suitable Semiconductor Device}\\
		S1& $i_{S_1} >0$& $v_{S_1} > 0$& FET\\
		S2& $i_{S_2}>0$ & $v_{S_2} < 0$  & Diode\\
		S3& $i_{S_3}>0$ & $v_{S_3} <0$ & Diode\\
	\end{tabular}
	\caption{Summary of the Proposed Semiconductors according to the catalog developed in the class}
\end{table}

\begin{figure}[!h]
	\centering
	\begin{circuitikz}[american] %, voltage shift=-1
		\ctikzset{inductors/scale=1.5, inductor=cute}
		\draw[thick] (0,0)to[V, l=$V_g$] (0,3) to [L, ,l=$L$, i>=$i_{L}$, 
		v=$v_{L}$](3,3);
		
		\draw[thick]
	 (3,1.5) node[nigfete] (nf) {};
\draw[thick] (nf.D) --(3,3);
\draw[thick] (nf.S) to[short, i>=$i_{S_1}$](3,0);
\draw[thick] (nf.G) node[anchor=east] {$G$};

		%\draw   (3,1.5) node[nigfete] (fet) {}
%		\draw (3,3) to[normal closed switch, l=$S_1$, i>=$i_{S_1}$] 
\draw[thick] (3,0) -- (0,0);
		\draw[thick] (3,3) to[capacitor,l=$C$, i>=$i_{C}$,v=$v_{C}$] (6,3);
		\draw[thick] (6,3) to[D, l=$S_2$,v=$v_{S_2}$, i>=$i_{S_2}$] (6,0) --(3,0);
		\draw[thick] (9,3) to[D, l=$S_3$, i>=$i_{S_3}$,v=$v_{S_3}$](6,3) ;
		\draw[thick] (9,3) to[capacitor,l=$C_o$, i>=$i_{C_o}$,v=$v_{C_o}$] (9,0)--(6,0);
		\draw[thick](9,3) --(12,3) to[R=$R_o$, v=$V_o$, i>=$I_{o}$](12,0) -- (9,0);
	\end{circuitikz}
	\caption{Final Switching Circuit with the Proposed Semiconductors: The control signal is supplied to the Gate terminal($G$) of the FET.}

\end{figure}

\end{document}
